\documentclass[
	a4paper,
	% showframes,
	% vline=2.2em,
	%maincolor=cvgreen,
	sidecolor=cyan,
	sectioncolor=materialblue,
	subsectioncolor=materialindigo,
	itemtextcolor=black!80,
	sidebarwidth=0.32\paperwidth,
	% topbottommargin=0.03\paperheight,
	% leftrightmargin=20pt,
	profilepicsize=3.5cm,
	% profilepicborderwidth=3.5pt,
	profilepicstyle=profilecircle,
	profilepiczoom=1.0,
	% profilepicxshift=0mm,
	% profilepicyshift=0mm,
	% profilepicrounding=1.0cm,
	% logowidth=4.5cm,
	% logospace=5mm,
	% logoposition=before,
]{CV}

% improve word spacing and hyphenation
\usepackage{microtype}
\usepackage{ragged2e}

% uncomment in case you don't want any hyphenation
% \usepackage[none]{hyphenat}

% take care of proper font encoding
\ifxetexorluatex
	\usepackage{fontspec}
	\defaultfontfeatures{Ligatures=TeX}
%	\newfontfamily\headingfont[Path = fonts/]{segoeuib.ttf} % local font
\else
	\usepackage[utf8]{inputenc}
	\usepackage[T1]{fontenc}
%	\usepackage[sfdefault]{noto} % use noto google font
\fi
\usepackage{amssymb}

% bubble diagram configuration

%-------------------------------------------------------------------------------
%                            PERSONAL INFORMATION
%-------------------------------------------------------------------------------
%% mandatory information
% your name
\cvname{Simon Jakobsson}
% job title/career



%% optional information
% profile picture
%\cvprofilepic{pictures/Simon.jpg}
% logo picture
%\cvlogopic{pics/logo_txt.png}

% NOTE: ordering in sidebar will mimic the following order
% date of birth

\cvbirthday{1996-03-25}
% short address/location, use \newline if more than 1 line is required
\cvaddress{Studievägen 11b, Linköping}
% phone number
\cvphone{+46 760171013}
% personal website
\cvlinkedin{linkedin.com/in/simon-jakobsson-871252144/}
%\cvgithub{github.com/Znarvl}
\cvsite{znarvl.github.io/portfolio/}
% email address
\cvmail{simme.jakobsson@gmail.com}
% pgp key
\cvcar{Driving License Category B}


%-------------------------------------------------------------------------------
%                              SIDEBAR 1st PAGE
%-------------------------------------------------------------------------------
% add more profile sections to sidebar on first page
\addtofrontsidebar{
	% include gosquare national flags from https://github.com/gosquared/flags;
	% naming according to ISO 3166-1 alpha-2 country codes

	\profilesection{Languages}
	\skill{\flag{pictures/Sweden.jpeg}}{Swedish - Mother tongue}
    \skill{\flag{pictures/Great-Britain.png}}{English - Professional proficient}
	\skill{\flag{pictures/Germany.png}}{German - Limited professional knowledge}
    \skill{\flag{pictures/Spain.png}}{Spanish - Limited professional knowledge}
    \profilesection{Programming Languages}
    \begin{itemize}
        \item Python
        \item JS/TS
        \item Java/Kotlin
        \item C
        \item C++
        \item R
        \item SQL
    \end{itemize}

    \profilesection{Tools and Frameworks}
    \begin{itemize}
        \item React
        \item Vue
        \item Node.js
        \item Pandas
        \item Numpy
        \item Slurm
        \item Django
        \item Postgres
        \item Git
        \item Docker
        \item Scikit-Learn
        \item Windows / Linux / MacOS
    \end{itemize}
}

    

%-------------------------------------------------------------------------------
%                         TABLE ENTRIES RIGHT COLUMN
%-------------------------------------------------------------------------------
\begin{document}

\makefrontsidebar

\cvsection{Utbildning}
\cvsubsection{Master}
\begin{cvtable}[1.5]
	\cvitem{2018 - 2023}{Civilingenjör Mjukvaruteknik}{Linköpings Universitet}
		{Studerade Mjukvaruteknik med inriktning mot utveckling av storskalig mjukvara.}
\end{cvtable}
\cvsubsection{Kandidatexamen}
\begin{cvtable}[1.5]
	\cvitem{2017 - 2018}{Systemvetenskap}{Lunds Universitet}
		{Studerade systemvetenskap vid Lunds Universitet under en period på sex månader innan jag bytade till Civilingenjör. }
\end{cvtable}
\cvsubsection{Gymnasieutbildning}
\begin{cvtable}[1.5]
	\cvitem{2012 - 2015}{Social science program}{Bladins}
		{Studerade social science program vid Bladins med specialiserad inriktning på Internationella Relationer. Under programmet hade jag möjlighet att studera alla ämnen på engelska.}
\end{cvtable}

\cvsection{Arbetserfarenhet}
\begin{cvtable}[3]
	\cvitem{Jun 2023 - Aug 2023}{Fullstack-utveckling}{Maxar Technologies}{Sommarjobb på Maxar Technologies där jag fick arbeta med (Django och ReactJS) }

	\cvitem{Sommar 2017 - 2021}{Trädgårdsassistent}{Nova trädgård}{Erbjöd support som trädgårdsassistent och var aktivt engagerad i uppgifter som plantering, bevattning och installation av bevattningssystem.}
	
    \cvitem{Feb 2016 - Aug 2016}{Lagerarbetare}{Personalcentralen}{
		Arbetade som lagerarbetare och erbjöd konsulttjänster till flera företag i Malmö-regionen, inklusive Lantmännen och Panduro Hobby.}

	\cvitem{Sep 2015 -- Feb 2016}{F2F Försäljning}{Pepperminds}{Arbetade som försäljare av prenumerationer för organisationer som Sightsavers och Hjärnfonden.}
\end{cvtable}


\cvsection{Extra-Curricular Aktiviteter}
\begin{cvtable}[3]
    \cvitem{2022- 2023}{Mottagningsansvarig vid sektionen för datateknik (STABEN)}{Linköpings Universitet}{Ansvarig för att hantera mottagningen vid sektionen för datateknik på Linköpings Universitet.}

    \cvitem{2020 - 2021}{Styrelsemedlem i sektionen för datateknik}{Linköpings Universitet}{
		Jag bidrog aktivt till tillväxten och utvecklingen av organisationen. Jag samarbetade med övriga styrelsemedlemmar för att implementera strategiska initiativ, organisera evenemang och skapa en stöttande miljö för studenterna under pandemin.}

	\cvitem{2019- 2020}{Orkesterledare för Datateknikens Orkester (D-band)}{Linköpings Universitet}{Jag hade en avgörande roll i att leda och motivera orkestermedlemmarna. Jag samordnade effektivt med arrangörer för att säkerställa smidig kommunikation och samarbete för att skapa den bästa möjliga framträdandet.}
\end{cvtable}

\end{document}